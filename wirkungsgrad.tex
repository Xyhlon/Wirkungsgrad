%! TeX program = lualatex
%---------------------------ALLGEMEINE IMPORTS-------------------------------------
\documentclass[12pt,english,ngerman]{scrartcl}
\input{./protokoll_template/template.latex/input/shared_preamble.tex}

% Kopfzeile
\ihead{WS22\\
	09.12.2022} \chead{\textsc{Stark} Matthias - 12004907 \\
	\textsc{Philipp} Maximilian - 11839611}
\ohead{FLAB 2 \\
	Leistungsmessung}
% Fußzeile
% \addbibresource{leistung.bib}
\usepackage{luacode}

\DeclareSIUnit\px{px}
\DeclareSIUnit\strich{|||}
\DeclareSIUnit\Var{var}
\DeclareSIUnit\VA{VA}

\begin{document}

\begin{luacode*}
	luatexbase.add_to_callback("wrapup_run", function()
	local outfile=io.open("20230308_20A_Philipp_Stark_WIRK_01.pdf", "wb")
	outfile:write(io.open(tex.jobname..".pdf", "rb"):read("a"))
	outfile:close()
	end, "final callback to rename pdf file")
\end{luacode*}

% \includepdf{./deckblatt.pdf}
\tableofcontents
\newpage

\section{Aufgabenstellung\label{Auf}}

\begin{itemize}
	\item Leistungsmessung einer ohmschen Last in einem Wechselstromkreis
	\item Wirkleistungsmessung im Drehstromnetz bei einer symmetrischen ohmschen Last in
	      Stern- und Dreiecksschaltung mit Aronschaltung
	\item Wirk- und Blindleistungsmessung bei einer allgemeinen Last im Dreiphasennetz
	\item Bauen eines rudimentärern Asynchron-Drehstrommotors
\end{itemize}

\section{Grundlagen}\label{Grund}

%TODO Matthias verbessern und bilder bitte 
\subsection{Solarzelle}
Eine Solarzelle, die auch als photovoltaische Zelle bezeichnet wird, wandelt
Sonnenlicht durch den Photoeffekt in elektrische Energie um. Der Photoeffekt
ist das Phänomen, bei dem ein Material Photonen (Lichtteilchen) absorbiert und
freie Elektronen und Löcher erzeugt, die zur Erzeugung von elektrischem Strom
genutzt werden können.

Die Grundstruktur einer Solarzelle besteht aus einer dünnen Halbleiterschicht,
die in der Regel aus Silizium besteht. Diese Halbleiterschicht befindet sich
zwischen zwei Metallkontakten, von denen einer auf der Ober- und einer auf der
Unterseite liegt. Wenn Sonnenlicht auf die Halbleiterschicht fällt, werden die
Photonen des Sonnenlichts von dem Halbleitermaterial absorbiert. Dies führt
dazu, dass einige der Elektronen im Material angeregt werden und auf ein
höheres Energieniveau springen, wobei ein "Loch" im Material zurückbleibt, wo
das Elektron vorher war.

Die angeregten Elektronen und die durch die absorbierten Photonen entstandenen
Löcher werden dann durch das elektrische Feld getrennt, das durch den Übergang
zwischen den beiden Schichten des Halbleiters entsteht. Dieses elektrische Feld
wird durch Dotierung des Halbleiters mit Verunreinigungen erzeugt, um einen
p-n-Übergang zu schaffen. Die p-Typ-Halbleiterschicht hat einen Überschuss an
positiv geladenen Löchern, während die n-Typ-Schicht einen Überschuss an
negativ geladenen Elektronen hat.

Die angeregten Elektronen werden in Richtung der n-Typ-Schicht gezogen, während
die Löcher in Richtung der p-Typ-Schicht gezogen werden. Durch diese
Ladungstrennung entsteht eine Spannungsdifferenz über der Halbleiterschicht,
die zur Erzeugung elektrischer Energie genutzt werden kann.

Die Shockley-Gleichung beschreibt die Beziehung zwischen dem Stromdurchsatz
einer Diode und der an ihr anliegenden Spannung. Da eine Solarzelle nichts
anderes als eine Diode ist, gilt dieses Gesetz auch in diesem Fall. Die
Shockley-Gleichung kann verwendet werden, um die Diodenkennlinie einer
Solarzelle anzupassen und wichtige Parameter wie den Fotostrom, den
Sättigungsrückstrom und den Idealitätsfaktor zu ermitteln. Diese Informationen
sind nützlich, um die Leistung einer Solarzelle zu bewerten und ihr Design für
eine maximale Effizienz zu optimieren.

Die Diodenkennlinie einer Solarzelle kann zur Bewertung und Charakterisierung
der Leistung der Zelle verwendet werden. Durch die Analyse der Diodenkennlinie
können die folgenden Schlüsselparameter der Solarzelle bestimmt werden:

\begin{enumerate}
	\item Leerlaufspannung (Voc): Dies ist die Spannung, bei der die Solarzelle keinen
	      Strom erzeugt, und entspricht der maximalen Ausgangsspannung der Zelle. Die
	      Leerlaufspannung wird durch die Beleuchtungsstärke und die intrinsischen
	      Eigenschaften der Zelle bestimmt.
	\item Kurzschlussstrom (Isc): Dies ist der Strom, der von der Solarzelle erzeugt
	      wird, wenn die Spannung über der Zelle gleich Null ist, und entspricht der
	      maximalen Stromabgabe der Zelle. Der Kurzschlussstrom wird durch die
	      Beleuchtungsstärke und die physikalischen Eigenschaften der Zelle bestimmt.
	\item Füllungsfaktor (FF): Er ist ein Maß für den Wirkungsgrad der Solarzelle und
	      wird durch das Verhältnis zwischen der maximalen Ausgangsleistung der Zelle und
	      dem Produkt aus Leerlaufspannung und Kurzschlussstrom bestimmt. Der Füllfaktor
	      wird von den physikalischen Eigenschaften der Zelle, wie der Rekombinationsrate
	      und dem Serien- und Nebenschlusswiderstand, beeinflusst.
	\item Maximaler Leistungspunkt (MPP): Dies ist der Punkt auf der Diodenkennlinie, an
	      dem die Solarzelle die maximale Leistung abgibt. Der Punkt maximaler Leistung
	      wird durch den Schnittpunkt der Lastlinie mit der Diodenkennlinie bestimmt und
	      wird von der Beleuchtungsstärke und den physikalischen Eigenschaften der Zelle
	      beeinflusst.
\end{enumerate}

Durch die Auswertung und Analyse dieser Schlüsselparameter kann die
Diodenkennlinie wertvolle Erkenntnisse über die Leistung einer Solarzelle
liefern. Diese Informationen können dazu genutzt werden, das Design der
Solarzelle zu optimieren und ihren Wirkungsgrad und ihre Leistung zu
verbessern.

\subsection{Wärmepumpe}
Eine Wärmepumpe ist ein Gerät, das Wärmeenergie von einem Ort zum anderen
überträgt. Sie wird sowohl für Heiz- als auch für Kühlzwecke eingesetzt und
basiert auf den Prinzipien der Thermodynamik.

Im Kühlbetrieb nimmt eine Wärmepumpe Wärme aus dem Innenraum auf und und gibt
sie nach außen ab. Im Heizbetrieb arbeitet sie umgekehrt, indem sie der
Außenluft, dem Boden oder dem Wasser Wärme entzieht aus der Außenluft, dem
Boden oder dem Wasser entzieht und an die Innenräume abgibt. Dies macht eine
effiziente Möglichkeit, ein Gebäude zu heizen oder zu kühlen, da sie weniger
Energie zur Wärme zu übertragen als sie zu erzeugen.

Die Effizienz einer Wärmepumpe wird durch ihre Leistungszahl $\epsilon$ (COP)
gemessen. Die Leistungszahl ist das Verhältnis zwischen der übertragenen
Wärmemenge und der von der Wärmepumpe verbrauchten Energie. Eine höherer
Leistungszahl bedeutet eine effizientere Wärmepumpe an, da sie mehr
Wärmeenergie für die gleiche Energiemenge übertragen kann Energiemenge mehr
Wärmeenergie übertragen kann (d.\ h.\ mit einem COP von $\epsilon$ kann man
$\epsilon$ Einheiten an Wärmeenergie Energie für jede verbrauchte
Energieeinheit übertragen).

\begin{equation}
	\epsilon = \frac{\dot{Q}}{P_{el}}
	\label{eq:Leistungszahl}
\end{equation}

Wobei $\dot{Q}$ der durch die Wärmepumpe verursachte Wärmefluss ist und
$P_{el}$ die benötigte Leistung ist die Wärmepumpe zu betreiben. Nun lässt sich
der Gütegrad mittels der Leistungszahl definieren, sie ist die Kenngröße der
Güte indem sie das Verhältnis der erhaltene Leistungszahl $\epsilon$ zu der
theoretisch Maximalen $\epsilon_\text{max} = \frac{T_h}{T_h-T_k}$ angibt.

\begin{equation}
	\eta = \frac{\epsilon}{\epsilon_\text{max}}
	\label{eq:guetefaktor}
\end{equation}

Der Kältekreislauf ist ein thermodynamischer Prozess der von einer Wärmepumpe
durchgeführt wird. Dieser Prozess entzieht Wärme aus einer Umgebung mit
niedriger Temperatur und gibt sie an eine Umgebung mit hoher Temperatur ab. Der
Prozess wird in Kältesystemen wie Klimaanlagen und Kühlschränken verwendet.
Kühlschränke.

Der Kältekreislauf besteht aus vier Hauptstufen:

\begin{enumerate}
	\item Isentrope Verdichtung: Das Kältemittel wird durch einen Kompressor komprimiert,
	      wodurch Druck und Temperatur erhöht werden. Druck und Temperatur erhöht. Diese
	      Stufe wird durch eine Linie dargestellt, die von niedrigem Druck und niedriger
	      Enthalpie zu hohem Druck und hoher Enthalpie isentrop verläuft.
	\item Isobare Kondensation: Das unter hohem Druck stehende und auf hohe Temperatur
	      gebrachte Kältemittel wird dann in einem Verflüssiger abgekühlt, wo es Wärme
	      abgibt und zu einer Flüssigkeit kondensiert. Diese Stufe wird wird durch eine
	      horizontale Linie dargestellt, die von hoher Enthalpie zu niedriger Enthalpie
	      verläuft, während der Druck konstant bleibt.
	\item Isenthalpe Drosselung: Das flüssige Kältemittel wird dann durch ein
	      Expansionsventil geleitet, wo sein Druck verringert wird, wodurch es zu einem
	      Gas mit niedrigem Druck verdampft. Diese Stufe wird durch eine Linie
	      dargestellt, bei konstanter Enthalpie verläuft, während der Druck abnimmt.
	\item Isobare Verdampfung: Das Niederdruck-Kältemittel absorbiert Wärme beim
	      Verdampfer aus der Umgebung auf, die es kühlen soll, und kehrt zum Kompressor
	      zurück, um den Kreislauf erneut zu starten. Diese Phase wird durch eine Linie
	      dargestellt, die von niedriger Enthalpie zu hoher Enthalpie verläuft, während
	      der Druck konstant bleibt.
\end{enumerate}

\section{Versuchsanordnung}\label{sec:versuchsanordnung}

\subsection{ohmsche Last in Wechselstromkreis}

Um die ohmsche Last einer Glühlampe im Wechselstromkreis zu messen, wird
folgender Versuchsaufbau aus \autoref{fig:aufbau_ohm} realisiert.

\begin{figure}[H]
	\centering
	\caption[Realer Versuchsaufbau für die Messung einer ohmschen Last]{Realer
		Versuchsaufbau für die Messung einer ohmschen Last \\
		1 \(\dots\) Transformator                          \\
		2 \(\dots\) seriell geschaltetes Strommessgerät    \\
		3 \(\dots\) seriell geschaltetes Leistungsmessgerät mit parallelen Anschluss
		zum Verbraucher                                    \\
		4 \(\dots\) ohmscher Verbraucher (Glühlampe)       \\
		5 \(\dots\) parallel geschaltetes Spannungsmessgerät
	}\label{fig:aufbau_ohm}
\end{figure}

\subsection{Symmetrische Last in Dreiecksschaltung}

Um die Wirkleistung von symmetrischen Verbrauchern in einer Dreiecksschaltung
zu Messen, wird die Aronschaltung nach folgendem Schaltplan aus
\autoref{fig:aufbau1} realisiert. Der tatsächliche Versuchsaufbau ist in
\autoref{fig:aufbau1_echt} ersichtlich.

\begin{figure}[H]
	\begin{center}
	\end{center}
	\caption[Schaltplan für die Messung der Wirkleistung mit Aronschaltung für symmetrische
		Verbraucher in Dreiecksschaltung] {Schaltplan für die Messung der Wirkleistung
		mit Aronschaltung für symmetrische Verbraucher in $I_i$ \(\dots\) entsprechende
		Ströme gemessen mit entsprechenden Amperemeter A                                 \\
		$U_i$ \(\dots\) entsprechende Spannungen gemessen mit entsprechenden Voltmeter V \\
		$R_i$ \(\dots\) entsprechender Widerstand durch die jeweiligen Verbraucher       \\
		$P_i$ \(\dots\) Powermeter in Aronschaltung
	}\label{fig:aufbau1}
\end{figure}

\begin{figure}[H]
	\begin{center}
	\end{center}
	\caption[Realer Versuchsaufbau für die Messung der Wirkleistung mit Aronschaltung für
		symmetrische Verbraucher in Dreiecksschaltung] {Realer Versuchsaufbau für die
		Messung der Wirkleistung mit Aronschaltung für symmetrische Verbraucher in
		Dreiecksschaltung. (Bei den Kabeln wurde ein Farbschema eingehalten, um eine
		bessere Übersicht zu ermöglichen.)                                  \\
		1 \(\dots\) Versorgungsspannung ($L_1$ rot, $L_2$ blau, $L_3$ gelb) \\
		2 \(\dots\) seriell geschaltete Strommessgeräte                     \\
		3 \(\dots\) seriell geschaltete Leistungsmessgeräte mit parallelen Anschlüssen
		nach der Aronschaltung (grün)                                       \\
		4 \(\dots\) parallel geschaltete Spannungsmessgeräte über die entsprechenden
		Verbraucher (schwarz)                                               \\
		5 \(\dots\) symmetrisch verteilte ohmsche Verbraucher (Glühlampen)
	}
	\label{fig:aufbau1_echt}
\end{figure}

\subsection{Symmetrische Last in Sternschaltung}

Um die Wirkleistung von symmetrischen Verbrauchern in einer Sternschaltung zu
Messen, wird die Aronschaltung nach folgendem Schaltplan aus
\autoref{fig:aufbau2} realisiert. Der tatsächliche Versuchsaufbau ist in
\autoref{fig:aufbau2_echt} ersichtlich.

\begin{figure}[H]
	\begin{center}
	\end{center}
	\caption[Schaltplan für die Messung der Wirkleistung mit Aronschaltung für symmetrische
		Verbraucher in Sternschaltung] {Schaltplan für die Messung der Wirkleistung mit
		Aronschaltung für symmetrische Verbraucher in $I_i$ \(\dots\) entsprechende
		Ströme gemessen mit entsprechenden Amperemeter A                                 \\
		$U_i$ \(\dots\) entsprechende Spannungen gemessen mit entsprechenden Voltmeter V \\
		$R_i$ \(\dots\) entsprechender Widerstand durch die jeweiligen Verbraucher       \\
		$P_i$ \(\dots\) Powermeter in Aronschaltung
	}
	\label{fig:aufbau2}
\end{figure}

\begin{figure}[H]
	\begin{center}
	\end{center}
	\caption[Realer Versuchsaufbau für die Messung der Wirkleistung mit Aronschaltung für
		symmetrische Verbraucher in Sternschaltung] {Realer Versuchsaufbau für die
		Messung der Wirkleistung mit Aronschaltung für symmetrische Verbraucher in
		Sternschaltung. (Bei den Kabeln wurde ein Farbschema eingehalten, um eine
		bessere Übersicht zu ermöglichen.)                                  \\
		1 \(\dots\) Versorgungsspannung ($L_1$ rot, $L_2$ blau, $L_3$ gelb) \\
		2 \(\dots\) seriell geschaltete Strommessgeräte                     \\
		3 \(\dots\) seriell geschaltete Leistungsmessgeräte mit parallelen Anschlüssen
		nach der Aronschaltung (grün)                                       \\
		4 \(\dots\) parallel geschaltete Spannungsmessgeräte über die entsprechenden
		Verbraucher (schwarz)                                               \\
		5 \(\dots\) symmetrisch verteilte ohmsche Verbraucher (Glühlampen)  \\
		6 \(\dots\) Strommessgerät zwischen Sternpunkt und Neutralleiter (grau)
	}\label{fig:aufbau2_echt}
\end{figure}

\subsection{Asymmetrische Last in Sternschaltung}

Um eine asymmetrische Last zu erreichen, wird der Aufbau aus
\autoref{fig:aufbau2} herangezogen, mit dem Unterschied, dass die Glühlampen
nicht gleichmäßig auf die Leiter aufgeteilt werden. Die gewählte Konfiguration
ist in \autoref{fig:lampenasym} ersichtlich.

\begin{figure}[H]
	\begin{center}
	\end{center}
	\caption[Entsprechende Konfiguration für eine asymmetrische Verteilung der Last]
	{Entsprechende Konfiguration für eine asymmetrische Verteilung der Last mit
		folgenden Verteilungen auf den Strängen: \\
		$L_1$ \(\dots\) 1 x \SI[]{60}{\watt}     \\
		$L_2$ \(\dots\) 2 x \SI[]{75}{\watt}     \\
		$L_3$ \(\dots\) 1 x \SI[]{75}{\watt} und 2 x \SI[]{60}{\watt}
	}\label{fig:lampenasym}
\end{figure}

\subsection{Asymmetrische Last in Sternschaltung und simulierten Kabelbruch}

Um einen Kabelbruch zu simulieren, wird der Aufbau aus \autoref{fig:aufbau2}
herangezogen. Nun wird der Kontakt des Neutralleiters unterbrochen, indem das
graue Kabel, sichtbar in \autoref{fig:aufbau1_echt}, aus dem Strompfad des
Multimeters entfernt und in den Spannungsbereich gesteckt wird, um eine
Spannungsmessung zu ermöglichen.

\subsection{Wirkleistungsmessung}

Um die Wirkleistung von allgemeinen Verbrauchern in Sternschaltung zu
bestimmen, wird die Schaltung nach folgendem Schaltplan aus
\autoref{fig:aufbau3} aufgebaut. Der tatsächliche Versuchsaufbau ist in
\autoref{fig:aufbau3_echt} ersichtlich.

\begin{figure}[H]
	\begin{center}
	\end{center}
	\caption[Schaltplan für die Messung der Wirkleistung für allgemeine Verbraucher in
		Sternschaltung] {Schaltplan für die Messung der Wirkleistung für allgemeine
		$I_i$ \(\dots\) entsprechende Ströme gemessen mit entsprechenden Amperemeter A   \\
		$U_i$ \(\dots\) entsprechende Spannungen gemessen mit entsprechenden Voltmeter V \\
		$R_i$ \(\dots\) entsprechender Widerstand durch die jeweiligen Verbraucher       \\
		$P_i$ \(\dots\) Powermeter
	}
	\label{fig:aufbau3}
\end{figure}

\begin{figure}[H]
	\begin{center}
	\end{center}
	\caption[Realer Versuchsaufbau für die Messung der Wirkleistung für allgemeine
		Verbraucher in Sternschaltung] {Realer Versuchsaufbau für die Messung der
		Wirkleistung für allgemeine Verbraucher in Sternschaltung. (Bei den Kabeln
		wurde ein Farbschema eingehalten, um eine bessere Übersicht zu ermöglichen.) \\
		1 \(\dots\) Versorgungsspannung ($L_1$ rot, $L_2$ blau, $L_3$ gelb)          \\
		2 \(\dots\) seriell geschaltete Strommessgeräte                              \\
		3 \(\dots\) seriell geschaltete Leistungsmessgeräte mit parallelen Anschlüssen
		zum Neutralleiter (grün)                                                     \\
		4 \(\dots\) parallel geschaltete analoge Spannungsmessgeräte über die
		entsprechenden Verbraucher (schwarz)                                         \\
		5 \(\dots\) parallel geschaltete digitale Spannungsmessgeräte über die
		entsprechenden Verbraucher (schwarz/grün)                                    \\
		6 \(\dots\) Strommessgerät zwischen Sternpunkt und Neutralleiter (grau)      \\
		7 \(\dots\) Heizwiderstände                                                  \\
		8 \(\dots\) ohmscher Verbraucher                                             \\
		9 \(\dots\) Kapazität (Kondensator)                                          \\
		10 \(\dots\) Induktivität (Spule)                                            \\
		11 \(\dots\) 2. Kapazität für Bonusaufgabe

	}\label{fig:aufbau3_echt}
\end{figure}

Für die Bonusaufgabe werden folgende Änderungen vorgenommen:

\begin{itemize}
	\item $L_1$ bleibt unverändert (Heizwiderstand)
	\item $L_2$ Schaltung von einem Heizwiderstand und einem Kondensator mit parallel geschalteter Induktivität
	\item $L_3$ Schaltung von einem Heizwiderstand und einem Kondensator
\end{itemize}

\subsection{Blindleistungsmessung}

Um die Blindleistung eines allgemeinen Verbrauchers sichtbar zu machen, wird
nun die Schaltung nach folgendem Schaltplan aus \autoref{fig:aufbau4}
aufgebaut, indem die grünen Kabel der Powermeter aus \autoref{fig:aufbau3_echt}
entsprechend modifiziert werden.

\begin{figure}[H]
	\begin{center}
	\end{center}
	\caption[Schaltplan für die Messung der Blindleistung für allgemeine Verbraucher in
		Sternschaltung] {Schaltplan für die Messung der Blindleistung für allgemeine
		$I_i$ \(\dots\) entsprechende Ströme gemessen mit entsprechenden Amperemeter A   \\
		$U_i$ \(\dots\) entsprechende Spannungen gemessen mit entsprechenden Voltmeter V \\
		$R_i$ \(\dots\) entsprechender Widerstand durch die jeweiligen Verbraucher       \\
		$P_i$ \(\dots\) Powermeter
	}
	\label{fig:aufbau4}
\end{figure}

\newpage
\subsection{Bau eines rudimentären Asynchron-Drehstrommotors}

Um den Bau eines rudimentären Asynchron-Drehstrommotors zu realisieren, werden
3 Spulen mit Eisenkern wie in \autoref{fig:motor} um eine drehbar gelagerte
Metallscheibe aufgestellt. Die Spulen werden mit vorgeschalteten
Heizwiderständen an die Versorgungsspannung geschlossen.

\begin{figure}[H]
	\begin{center}
	\end{center}
	\caption[Drehzahlmessung des rudimentären Asynchron-Drehstrommotors]{ Drehzahlmessung
		des rudimentären Asynchron-Drehstrommotors.
	}\label{fig:motor}
\end{figure}

\newpage
\section{Geräteliste}\label{sec:geraeteliste}

Für den Versuch werden die in \autoref{tab:gerate} aufgelisteten Geräte
verwendet.

\begin{table}[H]
	\caption{Verwendete Geräte
	}
	\begin{tblr}{lllll}
		\textbf{Gerätetyp}  & \textbf{Hersteller} & \textbf{Typ}          & \textbf{Inventar-Nr} & \textbf{Anmerkung} \\
		Transformator       & Thalheimer          & LTS 606               & 0161469              &                    \\
		{Box  mit                                                                                                     \\ Versorgungsspannung} & & & F1 & \\
		Strommessgerät      & Norma               & analog                & {VII/1106/9                               \\ VII/1106/6 \\ VII/1106/3} & 3 x \\
		Spannungsmessgerät  & Norma               & analog                & {VII/1120/2                               \\ VII/1120/1 E3} & 3 x \\
		Leistungsmessgerät  & Norma               & analog                & F18 F19              & 2 x                \\
		Leistungsmessgerät  & {Chauvin                                                                                \\ Arnoux} & analog & C.A.505 & \\
		Multimeter          & UNI-T               & UT51                  &                      & 2 x                \\
		Multimeter          & METEX               & M-3610B               &                      &                    \\
		Glühbirnen          &                     & {3x \SI{60}{\watt}                                                \\ 3x \SI{75}{\watt}} & & \\
		Kondensator         &                     & \SI{12}{\micro\farad} & G5 V                 & 2 x                \\
		Spule               &                     &                       & P                    &                    \\
		Heizwiderstand      &                     & 3x \SI{230}{\volt}    & JFR/158/13           & 3 x                \\
		Bananenstecker      &                     &                       &                      &                    \\
		Spule mit Eisenkern &                     &                       &                      & 3 x                \\
		{Metallscheibe auf                                                                                            \\ Sockel} & Hancaner & DT2234C & & \\
	\end{tblr}\label{tab:gerate}
\end{table}
\newpage
\section{Versuchsdurchführung und Messergebnisse}\label{sec:versuchsdurchfuehrung_messergebnisse}

\subsection{ohmsche Last in Wechselstromkreis}

Um die Leistung der ohmschen Last im Wechselstromkreis zu messen, wird der
Verbraucher, der in diesem Fall einer \SI[]{75}{\watt} Glühbirne entspricht,
wie in \autoref{fig:aufbau_ohm} ersichtlich, in den Stromkreis geschlossen.
Dabei ist besonders darauf zu achten, dass die Geräte richtig in den Stromkreis
geschlossen sind. Bei einem negativen Zeigerausschlag müssen also die Kabel
vertauscht werden. Auch sollt bei den Geräten der richtige Messbereich
ausgewählt werden, um sicherzustellen, dass die Geräte nicht überlastet werden,
aber dennoch ein genaues Ergebnis anzeigen.

Nun wird mithilfe des Schleifkontakts des Netzgeräts eine Ausgangsspannung von
\SI[]{20.0(6)}{\volt} erzeugt und diese kontinuierlich erhöht, bis schließlich
ein Wert von \SI[]{240.0(12)}{\volt} erreicht ist.

Die entsprechenden Werte der Messgeräte werden abgelesen und in folgender
\autoref{tab:werte_ohm} aufgelistet.

Bei dem vom Leistungsmessungsgerät abgelesenen Wert ist dabei besonders darauf
zu achten, ob das Gerät über die Verbindung für \SI{1}{\ampere} oder
\SI{5}{\ampere} verwendet wird.

\begin{table}[H]
	\caption[Gemessene Werte bei der Variation der ohmschen Last]{Gemessene Werte bei der
		Variation der ohmschen Last       \\
		$U \dots$ gemessene Spannung in V \\
		$I \dots$ gemessener Strom in A   \\
		$P \dots$ gemessene Leistung in W
	}\label{tab:werte_ohm}
	\centering
\end{table}

\subsection{Symmetrische Last in Dreiecksschaltung}

Um die Leistung eines symmetrischen Verbrauchs bei einer Dreiecksschaltung zu
betrachten, wird ein Aufbau nach \autoref{fig:aufbau1} herangezogen. Dabei ist
darauf zu achten, dass die Glühlampen symmetrisch auf die Stränge verteilt
sind, sich also auf jedem jeweils eine mit \SI[]{75}{\watt} und eine mit
\SI[]{60}{\watt} befindet. Alle abgelesenen Messwerte der Messgeräte sind in
folgender \autoref{tab:werte_sym_dreieck} aufgelistet.

\begin{table}[H]
	\caption[Abgelesene Werte bei symmetrischer Belastung in Dreiecksschaltung] {Abgelesene
		Werte bei symmetrischer Belastung in Dreiecksschaltung                     \\
		$I_i \dots$ gemessener Strom am i-ten Strang in A                          \\
		$I_{31} \dots$ gemessener Strom zwischen Sternpunkt und Neutralleiter in A \\
		$U_{ij} \dots$ gemessene Spannung zwischen den Strängen i und j in V       \\
		$P_{i}^M \dots$ gemessene Wirkleistungen in W (für genaue Bezeichnung siehe \autoref{fig:aufbau1})
	}\label{tab:werte_sym_dreieck}
	\centering
\end{table}

\subsection{Symmetrische Last in Sternschaltung}

Nun wird die Schaltung insofern modifiziert, dass nun eine Sternschaltung
vorliegt, wie in \autoref{fig:aufbau2} sichtbar.

Der besseren Übersicht halber, sind alle abgelesenen Werte der Messgeräte für
die Sternschaltung in folgender \autoref{tab:wert_stern} aufgelistet.

\begin{table}[H]
	\caption[Abgelesene Werte bei Sternschaltung] {Abgelesene Werte bei Sternschaltung                                        \\
		1. Zeile \dots symmetrische Belastung                                      \\
		2. Zeile \dots asymmetrische Belastung                                     \\
		3. Zeile \dots asymmetrische Belastung mit simulierten Kabelbruch          \\
		$I_i \dots$ gemessener Strom am i-ten Strang in A                          \\
		$I_{31} \dots$ gemessener Strom zwischen Sternpunkt und Neutralleiter in A \\
		$U_{i} \dots$ gemessene Spannung am i-ten Strang in V                      \\
		$P_{i}^M \dots$ gemessene Wirkleistungen in W (für genaue Bezeichnung siehe \autoref{fig:aufbau2})
	}\label{tab:wert_stern}
	\centering
\end{table}

\subsection{Asymmetrische Last in Sternschaltung}\label{sec:vers_asy_stern_ohne}

Nun werden die einzelnen Stränge verschieden stark beansprucht, indem die
Glühlampen asymmetrisch auf die Stränge verteilt werden. Dabei wird, wie
bereits in \autoref{sec:versuchsanordnung} angeführt, folgende Konfiguration
verwirklicht:

\begin{itemize}
	\item $L_1$ \(\dots\) 1 x \SI[]{60}{\watt}
	\item $L_2$ \(\dots\) 2 x \SI[]{75}{\watt}
	\item $L_3$ \(\dots\) 1 x \SI[]{75}{\watt} und 2 x \SI[]{60}{\watt}
\end{itemize}

Alle abgelesenen Werte der Messgeräte sind, wie bereits erwähnt, in
\autoref{tab:wert_stern} in der 2. Zeile angefügt.

\subsection{Asymmetrische Last in Sternschaltung und simulierten Kabelbruch}\label{sec:vers_asy_stern_mit}

Um nun einen Kabelbruch zu simulieren, wird die Verbindung des Neutralleiters
unterbrochen. Zusätzlich wird nun auch der Spannungsabfall an jener Stelle
gemessen, indem das entsprechende Multimeter als Spannungsmessgerät
umfunktioniert wird. Die so abgelesenen Werte der Messgeräte sind in
\autoref{tab:wert_stern} in der 3. Zeile aufgelistet.

\subsection{Wirkleistungsmessung}
Um die Wirkleistung eines realen Verbrauchers zu bestimmen, werden auch
Kapazitäten und Induktivitäten, wie in \autoref{fig:aufbau3} sichtbar, in die
Schaltung integriert.

Es sind, der besseren übersicht halber, wieder alle erhaltenen Werte für die
nächsten Aufgaben in \autoref{tab:werte_wirkleistung} aufgelistet. Die
gemessenen Werte der realen Verbraucher sind dabei in der 1. Zeile sichtbar.

Nun werden die Außenleiter $L_2$ und $L_3$ vertauscht, wodurch die Werte, aus
der 2.Zeile der \autoref{tab:werte_wirkleistung} entstehen.

Im Rahmen der Bonusaufgabe wird die Schaltung leicht modifiziert, wie bereits
in \autoref{sec:versuchsanordnung} angeführt. Dabei ist darauf zu achten, dass
am 2. Strang eine Parallelschaltung von Kapazität und Induktivität vorliegt.

Alle abgelesenen Werte der Messgeräte sind in der 3. Zeile in
\autoref{tab:werte_wirkleistung} aufgelistet.

\begin{table}[H]
	\caption[Abgelesene Werte für die Bestimmung der Wirkleistung] {\footnotesize Abgelesene
		Werte für die Bestimmung der Wirkleistung, bei den Bezeichnungen der Bauteile
		ist zu beachten, dass sich Zeile 3 und 6 auf die Bonusaufgabe beziehen und
		daher andere Bauteile verwendet werden.                                                   \\
		1. Zeile \dots Aufbau für die Messung der Wirkleistung: eines realen
		Verbrauchers                                                                              \\
		2. Zeile \dots Aufbau für die Messung der Wirkleistung: Wirkleistung eines
		realen Verbrauchers mit vertauschten Außenleitern                                         \\
		3. Zeile \dots Aufbau für die Messung der Wirkleistung: Wirkleistung bei
		modifizierter Schaltung                                                                   \\
		4. Zeile \dots Aufbau für die Messung der Wirkleistung: Blindleistung eines
		realen Verbrauchers                                                                       \\
		5. Zeile \dots Aufbau für die Messung der Wirkleistung: Blindleistung eines
		realen Verbrauchers mit vertauschten Außenleitern                                         \\
		6. Zeile \dots Aufbau für die Messung der Wirkleistung: Blindleistung bei
		modifizierter Schaltung                                                                   \\
		$I_i \dots$ gemessener Strom am i-ten Strang in A                                         \\
		$I_{31} \dots$ gemessener Strom zwischen Sternpunkt und Neutralleiter in A                \\
		$U_{i} \dots$ gemessener Spannungsabfall am i-ten Bauteil in V nach \autoref{fig:aufbau3} \\
		$P_{i}^M \dots$ gemessene Wirkleistungen am i-ten Strang in W
	}\label{tab:werte_wirkleistung}
	\centering
\end{table}

\subsection{Blindleistungsmessung}

Um die Blindleistung der Schaltung messbar zu machen, müssen die parallelen
Verbindungen der Powermeter, nach \autoref{fig:aufbau4} umgebaut werden.

Alle abgelesenen Werte der Messgeräte sind in der 4. Zeile in
\autoref{tab:werte_wirkleistung} aufgelistet.

Nun werden die Außenleiter $L_2$ und $L_3$ erneut vertauscht, wodurch die Werte
aus Zeile 5, aus \autoref{tab:werte_wirkleistung} entstehen.

Im Rahmen der Bonusaufgabe wird auch die leicht modifizierte Schaltung mit der
Parallelschaltung von Kapazität und Induktivität am 2. Strang aufgebaut.

Alle abgelesenen Werte der Messgeräte sind in der 6. Zeile von
\autoref{tab:werte_wirkleistung} aufgelistet.

\subsection{Bau eines rudimentärern Asynchron-Drehstrommotors}

Beim Bau des Drehstrommotors ist darauf zu achten, dass die Spulen richtig in
den Stromkreis geschlossen sind, sodass die maximale Drehzahl erreicht werden
kann. Auch die Abstände der Eisenkerne sind durch Probieren so einzustellen,
dass ein möglichst ruhiger Umlauf der Metallscheibe garantiert wird und sind
nicht bei allen 3 Spulen gleich, da diese bezüglich der Anzahl an Wickelungen
und Drahtdicke leicht verschieden sind.

Die Anzahl der Umdrehungen wird dabei mithilfe eines digitalen Zählers
bestimmt, der anhand eines Laserstrahls die Markierung auf der Metallscheibe
wahrnimmt. Die maximale Drehzahl, die mithilfe des Aufbaus realisiert werden
konnte war \SI{1691(2)}{\per\minute} Umdrehungen.

\section{Auswertung}\label{sec:auswertung}

\section{Diskussion}\label{sec:diskussion}

\section{Zusammenfassung}\label{sec:zusammenfassung}

\newpage

% \printbibliography
\listoffigures
\listoftables
\end{document}
